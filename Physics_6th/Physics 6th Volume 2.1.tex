\documentclass[UTF8,a4paper,12pt,scheme=chinese]{ctexbook}

%\setlength{\textwidth}{550pt}
%\setlength{\hoffset}{-1.2in}
%\setlength{\voffset}{-1in}

\usepackage{amsmath}
\usepackage{textcomp} 
\usepackage{graphicx}
\usepackage{xcolor}
\usepackage{setspace}
\usepackage{subeqnarray}

\linespread{1.6}

\newcommand{\hlx}[2]{%
	\IfEqCase{#1}{%
		{1}{
			\colorbox{yellow!50}{$\displaystyle#2$}
		}
		{b}{
			\colorbox{yellow!50}{$\scriptstyle#2$}
		}
		{3}{
			\colorbox{yellow!50}{$\scriptscriptstyle#2$}
		}
	}
		% you can add more cases here as desired
	
}%


%\pagestyle{headings}
\ctexset{
	section={
		name={,},
		number=\arabic{chapter}-\arabic{section},
		format=\youyuan\large ,
		%format=\flushleft\bfseries,
		},
	subsection={
		name={,、},
		%numberformat=\sffamily,
		numberformat=\CJKfamily{zhsong}\bfseries,
		%format=\youyuan\large ,
		format=\normalsize\flushleft\CJKfamily{zhsong}\bfseries,
		number=\chinese{subsection},
		},
	subsubsection={
		name={,},
		format=\normalsize\flushleft\CJKfamily{zhsong}\bfseries,
		number=\chinese{subsubsection}、,
	},
}


\newcommand{\ud}{\mathrm{d}}
\newcommand{\lc}{\left(}
\newcommand{\rc}{\right)}

\newcommand{\hl}[1]{\colorbox{yellow}{#1}}
\newcommand{\hla}[1]{%
	\colorbox{yellow!50}{$\displaystyle#1$}}
\newcommand{\hlb}[1]{%
	\colorbox{yellow!50}{$\scriptstyle#1$}}
\newcommand{\hlc}[1]{%
	\colorbox{yellow!50}{$\scriptscriptstyle#1$}}

\makeatletter
\DeclareFontFamily{U}{tipa}{}
\DeclareFontShape{U}{tipa}{m}{n}{<->tipa10}{}
\newcommand{\arc@char}{{\usefont{U}{tipa}{m}{n}\symbol{62}}}%

\newcommand{\arc}[1]{\mathpalette\arc@arc{#1}}

\newcommand{\arc@arc}[2]{%
	\sbox0{$\m@th#1#2$}%
	\vbox{
		\hbox{\resizebox{\wd0}{\height}{\arc@char}}
		\nointerlineskip
		\box0
	}%
}
\everymath{\displaystyle}
\allowdisplaybreaks
\makeatother

\begin{document}
	\chapter{振动}
	\section{简谐振动、振幅、周期和频率、相位}
	\begin{enumerate}
		\item $XF$\\
		$\Phi = \omega t+\hla{\varphi}$\\
		$x=A\cos\Phi$\\
		$v=-A\omega\sin\Phi$
		
	\end{enumerate}
	\section{旋转矢量}
	\section{单摆和复摆}
	\section{简谐振动的能量}
	\begin{enumerate}
		\item $E$\\
		$E=E_k+E_p=\frac{1}{2}kA^2$\\
		$\displaystyle
		E_k=\frac{1}{2}mv^2=\frac{1}{2}m\omega^2A^2\sin^2(\omega t+\varphi)$\\
		$\displaystyle
		E_p=\frac{1}{2}kx^2=\frac{1}{2}kA^2\cos^2(\omega t+\varphi)$
	\end{enumerate}
	
	\section{简谐振动的合成}
	\begin{enumerate}
		\item $X_1X_2\rightarrow X$
		\item $X_1X_2\rightarrow A$\\
		$A=\sqrt{{A_1}^2+{A_2}^2+2A_1A_2\cos(\varphi_2-\varphi_1)}$
		\item $X_1X_2\rightarrow \varphi$\\
		$\displaystyle
		\tan \varphi=\frac{A_1\sin\varphi_1+A_2\sin\varphi_2}{A_1\cos\varphi_1+A_2\cos\varphi_2}$
		
	\end{enumerate}
	\newpage
	\chapter{波动}
	\section{机械波的几个概念}
	\section{平面简谐波的波函数}
	\subsection{平面简谐波的波函数}
	\section{波的能量、能流密度}
	\section{惠更斯原理、波的衍射和干涉}
	\subsection{惠更斯原理}
	\subsection{波的衍射}
	\subsection{波的干涉}
	\begin{enumerate}
		\item $yF$\\
		$\displaystyle
		y=A\cos\left[\omega\left(t\mp\frac{x-x_0}{\hla{u}}\right)+\varphi\right]\\
		y=A\cos\left[2\pi\left(\hla{\nu} t\mp\frac{x-x_0}{\hla{\lambda}}\right)+\varphi\right]
		$
		\item $\varphi_1\varphi_2\rightarrow A$\\
		$\displaystyle
		A=\sqrt{{A_1}^2+{A_2}^2+2A_1A_2\cos\left(\varphi_2-\varphi_1-2\pi\frac{r_2-r_1}{\lambda}\right)}$
		\item $\Delta \varphi \rightarrow \Delta x:$\\
		$\displaystyle
		y=A\cos \left[\omega(t-\frac{x}{u})+\varphi\right]\\
		\Phi = \omega(t-\frac{x}{u})+\varphi\\
		\Delta \Phi = \left[\omega(t-\frac{x_1}{u})+\varphi\right] - \left[\omega(t-\frac{x_2}{u})+\varphi\right]
		$\\
		同一列波初相相同,t相同\\
		$\displaystyle
		\Delta \Phi = \omega\left(\frac{x_1-x_2}{u}\right) = \frac{2\pi}{T}\left(\frac{x_1-x_2}{u}\right) = 2\pi\frac{x_1-x_2}{\lambda}
		$
	\end{enumerate}
	\chapter{光学}
	\section{相干光}
	\section{杨氏双缝干涉 劳埃德镜}
	\subsection{杨氏双缝干涉}
	\subsection{*杨氏双缝干涉的光强分布}
	\subsection{光程和光程差}
	\begin{itemize}
		\item 波程差$ \Delta r $
		\subitem 两相干光到达相遇点时的几何路程差
		\item 相位差$ \Delta\varphi$
		\subitem $ \Delta\varphi=\dfrac{2\pi}{\lambda}\Delta r $
		\item 光程
		\subitem 光在折射率为n的介质中通过几何路程L所发生的相位变化,相当于光在真空中通过nL的路程所发生的相位变化
		\subitem $ nL $
		\item 光程差 $ \Delta $
		\subitem $ \Delta\varphi=\dfrac{2\pi}{\lambda}\Delta $
		\subitem 加强:$ \Delta=\pm k\lambda ,\ k = 0,1,2$
		\subitem 减弱 $\Delta  =  \pm \left( {2k + 1} \right)\frac{\lambda }{2},k = 0,1,2$
	\end{itemize}
	\subsection{缝宽对干莎条纹的影响 空间相干性}
	\subsection{劳埃德镜}
	光从光速较大(折射率较小)的介质射向光速较小(折射率较大)的介质时,在本书所讨论的一些情况中,反射光的相位较之入射光的相位跃变了$ \pi $
	\paragraph{半波损失} 反射光与入射光之间附加了半个波长$ \dfrac{\lambda}{2} $的波程差
	\section{薄膜干涉}
	\subsection{薄膜干涉的光程差}
	\paragraph{反射光干涉条件}
	\begin{itemize}
		\item $ i $:入射角,与竖直方向夹角,垂直时$ \sin i=0 $
	\end{itemize}
	$${\Delta _r} = 2d\sqrt {n_2^2 - n_1^2{{\sin }^2}{i}}  + \frac{\lambda }{2} = \left\lbrace {\begin{array}{*{20}{c}}
		{k\lambda}\quad k=1,2,\cdots\mbox{加强}\\
		{\left( {2k + 1} \right)\frac{\lambda }{2}\quad k=0,1,2,\cdots\mbox{减弱}}
		\end{array}} \right.$$
	\paragraph{透射光干涉条件}
	$${\Delta _t} = 2{\rm{d}}\sqrt {n_2^2 - n_1^2{{\sin }^2}i} $$
	\paragraph{使用透镜并不引起附加的光程差。}
	\subsection{*等倾干涉}
	\section{劈尖 牛顿环 迈克耳孙干涉仪}
	\subsection{劈尖}
	
	
	\chapter{气体动理论}
	\section{平衡态}
	\subsection{理想气体物态方程}
	\begin{enumerate}
		\item $pV=\hla{N}kT$:
		\subitem $k$:玻尔兹曼常量($\mathrm{1.38\times10^{-23}J\cdot K^{-1}}$)
		\subitem $N$:分子数量(个)
		\item $pV=\nu RT$
		\subitem $\nu = \frac{N}{N_A}$:物质的量(mol)
		\subitem $R=N_Ak$:摩尔气体常量($\mathrm{8.31\mathrm{J}/(mol\cdot K)}$)
		\item $pV=\frac{m'}{M}RT$
		\subitem $m'=mN$:气体的总质量(kg)
		\subitem $M=mN_A$:气体的摩尔质量(g/mol)
		\subitem $\frac{m'}{M}=\frac{mN}{mN_A}=\frac{N}{N_A}=\nu$
		\subitem $\hla{\rho=\frac{m'}{V}=\frac{pM}{RT}}$
		\subitem $m=\frac{M}{N_A}$
		\item $\hla{p=nkT}$
		\subitem $\hla{n=\frac{N}{V}}$:\bfseries{分子数密度} (单位体积的分子数)
		\subitem $n=\frac{p}{kT}$
	\end{enumerate}
	\section{物质的微观模型、统计规律性}
	\section{理想气体的压强公式}
	\subsection{理想气体的微观模型}
	\begin{enumerate}
		\item 分子线度与分子间距相比较可忽略:质点
		\item 除碰撞外,分子间无相互作用:自由质点,没有势能
		\item 碰撞为弹性碰撞:弹性质点
	\end{enumerate}
	\subsection{(平衡态)理想气体的统计假设}
	\begin{enumerate}
		\item 分子数密度处处相等(均匀分布)
		\item 分子沿各个方向运动概率相同
		\subitem 任意时刻向各个方向运动的分子数相同
		\subitem 分子速度在各个方向分量的各种平均值相等
		\subsubitem $\overline{v_x}=\overline{v_y}=\overline{v_z}$
		\subsubitem $\overline{v^2_x}=\overline{v^2_y}=\overline{v^2_z}$
	\end{enumerate}
	\subsection{(平衡态)理想气体的压强公式}
	\paragraph{推导压强公式的出发点}
	\begin{enumerate}
		\item 气体压强是大量分子不断碰撞容器壁的结果
		\item 压强等于单位时间内容器壁上单位面积所受的平均冲量
		\item 个别分子服从经典力学定律
		\item 大量分子整体服从统计规律
	\end{enumerate}
	\paragraph{压强公式}
	\begin{enumerate}
		\item 一个分子i与容器壁A碰撞一次的冲量\\
		$I=\Delta mv=2mv_{ix}$
		\subitem $I' = -mv_{ix}-mv_{ix}=-2mv_{ix}$:对i分子冲量
		\item 一个分子i与容器壁A碰撞一秒的次数\\
		$f=\frac{1}{T}=\frac{v_x}{2x}$
		\subitem $T=\frac{s}{v}=\frac{2x}{v_{ix}}$ :碰撞一次的间隔
		\item 一个分子i与容器壁A碰撞一秒的冲量\\
		$I=I\cdot f= 2mv_{ix} \cdot \frac{v_{ix}}{2x}=\frac{mv^2_{ix}}{x}$
		\item 一个分子i与容器壁A碰撞的冲力\\
		$F_i=\frac{I}{\Delta t}=\frac{\frac{mv^2_{ix}}{x}}{1}=\frac{mv^2_{ix}}{x}$
		\item N个分子i与容器壁A碰撞的冲力\\
		$\overline{F}=\sum_{i=1}^{N}F_i=\sum_{i=1}^{N}\frac{mv^2_{ix}}{x}$
		\item N个分子i与容器壁A碰撞的压强\\
		$\displaystyle
		p
		=\frac{\overline{F}}{S}
		=\frac{\sum_{i=1}^{N}\frac{mv^2_{ix}}{x}}{yz}
		=\sum_{i=1}^{N}\frac{mv^2_{ix}}{xyz}
		=\frac{m}{V}\sum_{i=1}^{N}v^2_{ix}\\
		=m\frac{N}{V}\frac{\sum_{i=1}^{N}v^2_{ix}}{N}\mbox{(上下同乘N)}\\
		=m\hla{n\overline{v^2_{ix}}}
		$
		\subitem $N$:总分子数
		\subitem $n = \frac{N}{V}$:分子数密度(单位体积的分子数)
		\subitem $\overline{v^2_{ix}} = \frac{\sum_{i=1}^{N}v^2_{ix}}{N} $:速度分量平方的平均值
		\item N个分子i与容器壁A碰撞的压强与分子总速度的关系\\
		$p = mn\overline{v^2_{ix}} = \hla{\frac{1}{3}mn\overline{v^2}}$\\
		$p=\frac{2}{3}n(\frac{1}{2}m\overline{v^2})=\hla{\frac{2}{3}n\overline{\varepsilon_k}}$
		\subitem $\overline{v^2} = 3\overline{v^2_{ix}} $:\qquad 总速率平方的平均值
		\subitem $\hla{\overline{\varepsilon_k}} = \frac{1}{2}m\overline{v^2}$:\quad (单个)分子平均\hl{平动}动能
	\end{enumerate}
	\paragraph{压强公式的物理意义}~{}\\
	压强具有统计意义\\
	将宏观量$p$和微观量$\overline{\varepsilon_k}$联系起来
	\section{理想气体分子的平均平动动能与温度的关系}
	\subsection{温度的微观解释}
	\paragraph{理想气体的温度公式}
	\begin{itemize}
		\item 理想气体压强公式 $p=\frac{2}{3}n\overline{\varepsilon_k}$
		\item 理想气体物态方程 $p=nkT$
		\item 分子平均平动动能 $\frac{2}{3}n\overline{\varepsilon_k} = nkT\\
		\hla{\overline{\varepsilon_k} = \frac{3}{2}kT}$
	\end{itemize}
	\paragraph{理想气体的物理意义}
	\begin{enumerate}
		\item 温度是分子平均平动动能的量度:统计平均值
		\item 温度是大量分子的集体表现
		\item 在同一温度下各种气体分子平均平动动能均相等
	\end{enumerate}
	\section{能量均分定理、理想气体的内能}
	\subsection{自由度}
	\subsection{能量均分定理}
	\paragraph{(单个)分子平均总动能(平动动能+转动动能)}
	\begin{enumerate}
		\item 通式$\overline{\varepsilon}=\frac{i}{2}kT$
		\item 单原子分子 $\overline{\varepsilon}=\frac{3}{2}kT=
		\qquad\hla{\overline{\varepsilon_k}}$
		\item 双原子分子 $\overline{\varepsilon}=\frac{5}{2}kT
		\qquad=
		\overbrace{\frac{3}{2}kT}
		^{\mbox{已推导}}
		+\frac{2}{2}kT
		$
		\item 多原子分子 $\overline{\varepsilon}=\frac{6}{2}kT
		\qquad=
		\overbrace{\frac{3}{2}kT}
		^{\mbox{已推导}}
		+\frac{3}{2}kT
		$
	\end{enumerate}
	\subsection{理想气体的内能}
	\paragraph{模型}
	\paragraph{刚性分子理想气体内能}
	\begin{enumerate}
		\item $E_{mol} = 1N_A\cdot\overline{\varepsilon} = 1N_A\cdot\frac{i}{2}kT = 1\cdot\frac{i}{2}RT$:\qquad 1mol理想气体内能
		\item $E_{\nu mol} = \nu N_A\cdot\overline{\varepsilon} = \nu N_A\cdot\frac{i}{2}kT = \hla{\nu\frac{i}{2}RT} = \frac{m'}{M}\frac{i}{2}RT$:\qquad $\nu$mol理想气体内能
	\end{enumerate}
	\paragraph{内能的改变量}
	\begin{itemize}
		\item $\Delta E = \frac{m'}{M}\frac{i}{2}R\Delta T$
	\end{itemize}
	\chapter{热力学基础}
	\section{准静态过程\quad 功\quad 热量}
	\subsection{准静态过程}
	\subsection{功}
	$$ W=Fl=PSl=pV $$
	$$ W=\int_{V_1}^{V_2}p\ud V $$
	\subsection{热量}
	\paragraph{定义}系统与外界直接由于存在温度差而传递的能量
	\paragraph{字母}$ Q $
	\paragraph{类型}过程量
	\section{热力学第一定律\quad 内能}
	\subsection{内能}
	\paragraph{定义}系统处于某状态而具有的能量
	\paragraph{字母}$ E $
	\subsection{热力学第一定律}
	\paragraph{能量守恒定律}
	$$ E_{\mbox{\zihao{7}(末)}}-E_{0\mbox{\zihao{7}(初)}} = 
	\overbrace{Q}
	^{\mbox{\zihao{6}外界向系统传递的热量}}
	+
	\overbrace{W^{ex}}
	^{\mbox{\zihao{6}外界对系统做功}}
	$$
	$$
	W^{ex}=
	-\overbrace{W}
	^{\mbox{\zihao{6}系统对外界做功}}
	$$
	\paragraph{热力学第一定律}
	$$
	\overbrace{Q}
	^{\mbox{\zihao{6}外界向系统传递的热量}}
	=
	\overbrace{W}
	^{\mbox{\zihao{6}系统对外界做功}}
	+
	\overbrace{E-E_0}
	^{\mbox{\zihao{6}系统内能变化}}
	=W+\Delta E 
	$$
	$$ \ud Q = \ud W + \ud E$$
	\section{理想气体的等体过程和等压过程\quad 摩尔热容}
	\subsection{等体过程\quad 摩尔定容热容}
%	\begin{spacing}{1.0}
\paragraph{原始公式}
	\begin{eqnarray}
	W&=&\int_{V_1}^{V_2}p\ud V\\
	\ud Q&=&
	\overbrace{\ud W}
	^{
		\parbox[t][0.5cm]{1.1cm}{
			\centering
			\begin{spacing}{1}
			\zihao{6}
			等体过程\\此项为0
			\end{spacing}
	}
	}
	+\ud E\\
	Q_V&=&
	\ud E\\
	Q_{V,m}&=&
	\ud E_m\\
	\dfrac{\ud Q_{V,m}}{\ud T} &=& 
	\dfrac{\ud E_m}{\ud T}
	\end{eqnarray}
	\paragraph{对等式右边展开}
	\begin{subeqnarray}	
	C_{V,m} &=&\dfrac{\ud E_m}{\ud T}\\
	C_{V,m}\ud T &=& \ud E_m\\
	\hla{\nu C_{V,m}\ud T} &=& \hla{\ud E}\\
	C_{V,m}\ud T	&=&
	\overbrace{\frac{i}{2}RT}
	^{
		\parbox[t][0.5cm]{1.5cm}{
			\begin{spacing}{1}
				\centering
				\zihao{6}
				单个分子\\内能公式
			\end{spacing}
		}
	}
	\\
	C_{V,m}	&=&\hla{\frac{i}{2}R}
	\end{subeqnarray}	
	\paragraph{对等式左边展开}
	\begin{subeqnarray}
	C_{V,m}&=&\dfrac{\ud Q_{V,m}}{\ud T}\\
	C_{V,m}\ud T&=&\ud Q_{V,m}\\
	C_{V,m}\ud T&=&\dfrac{\ud Q_{V}}{\nu}\\
	\nu C_{V,m}\ud T&=&\ud Q_{V}\\
	\hla{\nu C_{V}(T_2-T_1)}&=&Q_{V}
	\end{subeqnarray}
	\begin{subeqnarray}
		\nu C_{V}\ud T&=&\ud E\\
	\end{subeqnarray}
\subsection{等压过程\quad 摩尔定压热容}
\paragraph{原始公式}
	\begin{eqnarray}
		\ud Q &=& \ud W + \ud E\\
		\ud Q_P &=& \ud W + \ud E\\
		\ud Q_P &=& p\ud V + \ud E\\
		\ud Q_{P,m} &=& p\ud V + \ud E_m\\
		\dfrac{\ud Q_{P,m}}{\ud T} &=& \dfrac{p\ud V + \ud E_m}{\ud T}
	\end{eqnarray}	
	\paragraph{对等式右边展开}
	\begin{subeqnarray}
		C_{p,m} &=& \dfrac{
			\overbrace{p}^{
				\mbox{\zihao{6}为\hl{常数}}
			}
			\ud V}{\ud T}+\dfrac{\ud E_m}{\ud T}\\
		\
		C_{p,m} &=& 
		\overbrace{R}^{
			\parbox[t][0.8cm]{1.6cm}{
				\begin{spacing}{1}
				\zihao{6}
				$pV=RT$
				\\
				两边取微分
				\\
				$p\ud V = R \ud T$
				\end{spacing}
			}
		}
		 + C_{V,m}\\
		C_{p,m} &=& R + \dfrac{i}{2}R\\
		C_{p,m} &=& \hla{\dfrac{i+2}{2}R}
	\end{subeqnarray}
	\paragraph{对等式左边展开}
	\begin{subeqnarray}	
		C_{p,m} &=& \dfrac{\ud Q_{p,m}}{\ud T}\\
		C_{p,m}\ud T &=& \ud Q_{p,m}\\
		C_{p,m}\ud T&=&\dfrac{\ud Q_{p}}{\nu}\\
		\nu C_{p,m}\ud T&=&\ud Q_{p}\\
		\hla{\nu C_{P}(T_2-T_1)}&=&Q_{p}
	\end{subeqnarray}
	\begin{subeqnarray}	
		Q_P &=& \int_{V_1}^{V_2}p\ud V + \Delta E\\
		Q_P &=& p(V_2-V_1) + E_2-E_1\\
		Q_P &=& p(V_2-V_1) + \nu C_{V,m}(T_2-T_1)\\
		Q_P &=& \nu C_{p,m}(T_2-T_1)
	\end{subeqnarray}
%\end{spacing}
	\section{理想气体的等温过程和绝热过程}
	\subsection{等温过程}
	\paragraph{原始公式}
	\begin{eqnarray}
	\ud Q &=& \ud W + 
	\overbrace{\ud E}
	^{
		\parbox[t][0.5cm]{1.1cm}{
			\centering
			\begin{spacing}{1}
			\zihao{6}
			等温过程\\此项为0
			\end{spacing}
		}
	}\\
	\ud Q_T &=& \ud W_T\\
	\ud W_T &=& p\ud V\\
	\ud W_T &=& 
	\overbrace{\dfrac{\nu RT}{V}}
	^{
		\parbox[t][0.2cm]{1.1cm}{
			\centering
			\begin{spacing}{1}
			\zihao{6}
			状态方程
			\end{spacing}
		}
	}
	\ud V\\
	\int_{W_1}^{W_2}\ud W_T &=& \int_{V_1}^{V_2}
	\dfrac{\nu RT}{V}
	\ud V\\
	W_T &=& \nu RT\int_{V_1}^{V_2}\dfrac{1}{V}\ud V\\
	W_T &=& \nu RT\ln\dfrac{V_2}{V_1}\\
	W_T &=& \nu RT\ln\dfrac{p_1}{p_2}\\
	\end{eqnarray}
	\subsection{绝热过程}
	\paragraph{前提公式}
	\begin{eqnarray}
	pV &=& \nu RT\\
	\ud p \cdot V + p\ud V &=& \nu R\ud T\\
	C_{V,m}\ud p \cdot V + C_{V,m}p\ud V &=& C_{V,m}\nu R\ud T
	\end{eqnarray}
	\paragraph{原始公式}
	\begin{eqnarray}
	\overbrace{\ud Q}
	^{
		\parbox[t][0.5cm]{1.1cm}{
			\centering
			\begin{spacing}{1}
			\zihao{6}
			绝热过程\\此项为0
			\end{spacing}
		}
	}
	&=& \ud W + \ud E\\
	0 &=& \ud W_a + \ud E\\
	0 &=& p\ud V + \nu C_{V,m}\ud T(6c)\\
	\nu C_{V,m}\ud T &=& -p\ud V
	\end{eqnarray}
	\paragraph{代入前提公式}
	\begin{eqnarray}
	C_{V,m}\ud p \cdot V + C_{V,m}p\ud V &=& C_{V,m}\nu R\ud T\\
	C_{V,m}\ud p \cdot V + C_{V,m}p\ud V &=& -pR\ud V\\	
	C_{V,m}\ud p \cdot V  &=& - C_{V,m}p\ud V+pR\ud V\\	
	C_{V,m}\ud p \cdot V  &=& - (C_{V,m}+R)p\ud V\\	
	\dfrac{C_{V,m}}{(C_{V,m}+R)}\dfrac{\ud p}{p}  &=& - \dfrac{\ud V}{V}\\	
	\dfrac{C_{V,m}}{(C_{p,m})}\dfrac{\ud p}{p}  &=& - \dfrac{\ud V}{V}\\	
	-\dfrac{\ud p}{p}  &=&  \gamma\dfrac{\ud V}{V}\\	
	-\ln p  &=&  \gamma\ln V+C\\	
	C  &=&  \gamma\ln V+\ln p\\	
	C  &=&  \ln V^\gamma+\ln p\\	
	C  &=&  \ln( V^\gamma p)\\	
	C_1  &=&   \hla{V^\gamma p}
	\end{eqnarray}
	\begin{subeqnarray}	
		V^{\gamma-1}T &=& C_2\\
		p^{\gamma-1}T^{-\gamma} &=& C_3
	\end{subeqnarray}	
	
\end{document}
