\documentclass[UTF8,a4paper,12pt,scheme=chinese]{ctexbook}

%\setlength{\textwidth}{550pt}
%\setlength{\hoffset}{-1.2in}
%\setlength{\voffset}{-1in}

\usepackage{amsmath}
\usepackage{textcomp} 
\usepackage{graphicx}
\usepackage{xcolor}

%\linespread{1.6}

\newcommand{\hlx}[2]{%
	\IfEqCase{#1}{%
		{1}{
			\colorbox{yellow!50}{$\displaystyle#2$}
		}
		{b}{
			\colorbox{yellow!50}{$\scriptstyle#2$}
		}
		{3}{
			\colorbox{yellow!50}{$\scriptscriptstyle#2$}
		}
	}
		% you can add more cases here as desired
	
}%


%\pagestyle{headings}
\ctexset{
	section={
		name={,},
		numberformat=\sffamily,
		format=\youyuan\large ,
		number=\arabic{chapter}-\arabic{section},
		},
	subsection={
		name={,},
		format=\normalsize\flushleft\CJKfamily{zhsong}\bfseries,
		number=\chinese{subsection}、,
	},
}


\newcommand{\ud}{\mathrm{d}}
\newcommand{\lc}{\left(}
\newcommand{\rc}{\right)}

\newcommand{\hl}[1]{\colorbox{yellow}{#1}}
\newcommand{\hla}[1]{%
	\colorbox{yellow!50}{$\displaystyle#1$}}
\newcommand{\hlb}[1]{%
	\colorbox{yellow!50}{$\scriptstyle#1$}}
\newcommand{\hlc}[1]{%
	\colorbox{yellow!50}{$\scriptscriptstyle#1$}}

\makeatletter
\DeclareFontFamily{U}{tipa}{}
\DeclareFontShape{U}{tipa}{m}{n}{<->tipa10}{}
\newcommand{\arc@char}{{\usefont{U}{tipa}{m}{n}\symbol{62}}}%

\newcommand{\arc}[1]{\mathpalette\arc@arc{#1}}

\newcommand{\arc@arc}[2]{%
	\sbox0{$\m@th#1#2$}%
	\vbox{
		\hbox{\resizebox{\wd0}{\height}{\arc@char}}
		\nointerlineskip
		\box0
	}%
}
\makeatother

\begin{document}
	\begin{Large}
		\chapter{振动}
		\section{简谐振动、振幅、周期和频率、相位}
		\section{旋转矢量}
		\section{单摆和复摆}
		\section{简谐振动的能量}
		\section{简谐振动的合成}
		\chapter{波动}
		\section{机械波的几个概念}
		\section{平面简谐波的波函数}
		\subsection{平面简谐波的波函数}
		\section{波的能量、能流密度}
		\section{惠更斯原理、波的衍射和干涉}
		\subsection{惠更斯原理}
		\subsection{波的衍射}
		\subsection{波的干涉}
		$\displaystyle\\
		\Delta \varphi \rightarrow \Delta x:\\
		y=A\cos \left[\omega(t-\frac{x}{u})+\varphi\right]\\
		\Phi = \omega(t-\frac{x}{u})+\varphi\\
		\Delta \Phi = \left[\omega(t-\frac{x_1}{u})+\varphi\right] - \left[\omega(t-\frac{x_2}{u})+\varphi\right]
		$\\
		同一列波初相相同,t相同\\
		$\displaystyle
		\Delta \Phi = \omega\left(\frac{x_1-x_2}{u}\right) = \frac{2\pi}{T}\left(\frac{x_1-x_2}{u}\right) = 2\pi\frac{x_1-x_2}{\lambda}
		$
		\chapter{气体动理论}
	\end{Large}

\end{document}
