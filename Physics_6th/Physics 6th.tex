\documentclass[UTF8,a4paper,12pt,scheme=chinese]{ctexart}

\setlength{\textwidth}{550pt}
\setlength{\hoffset}{-1.2in}
\setlength{\voffset}{-1in}

\usepackage{amsmath}
\usepackage{textcomp} 
\usepackage{graphicx}
\usepackage{xcolor}

\linespread{1.6}

\newcommand{\hlx}[2]{%
	\IfEqCase{#1}{%
		{1}{
			\colorbox{yellow!50}{$\displaystyle#2$}
		}
		{b}{
			\colorbox{yellow!50}{$\scriptstyle#2$}
		}
		{3}{
			\colorbox{yellow!50}{$\scriptscriptstyle#2$}
		}
	}
		% you can add more cases here as desired
	
}%


\pagestyle{headings}
\ctexset{
	section={
		name={第,章},
		number=\arabic{section},
		},
	subsection={
		name={,节},
		number=\arabic{subsection}-\arabic{section},
		},
}


\newcommand{\ud}{\mathrm{d}}
\newcommand{\lc}{\left(}
\newcommand{\rc}{\right)}

\newcommand{\hl}[1]{\colorbox{yellow}{#1}}
\newcommand{\hla}[1]{%
	\colorbox{yellow!50}{$\displaystyle#1$}}
\newcommand{\hlb}[1]{%
	\colorbox{yellow!50}{$\scriptstyle#1$}}
\newcommand{\hlc}[1]{%
	\colorbox{yellow!50}{$\scriptscriptstyle#1$}}

\makeatletter
\DeclareFontFamily{U}{tipa}{}
\DeclareFontShape{U}{tipa}{m}{n}{<->tipa10}{}
\newcommand{\arc@char}{{\usefont{U}{tipa}{m}{n}\symbol{62}}}%

\newcommand{\arc}[1]{\mathpalette\arc@arc{#1}}

\newcommand{\arc@arc}[2]{%
	\sbox0{$\m@th#1#2$}%
	\vbox{
		\hbox{\resizebox{\wd0}{\height}{\arc@char}}
		\nointerlineskip
		\box0
	}%
}
\makeatother

\begin{document}
	\begin{Large}
		\begin{enumerate}
			\item 一半径为R、质量为m的均匀圆盘平放在粗糙的水平面上。若它的初速度为$w_0$,绕中心$O$旋转,问经过多长时间圆盘才停止。(设摩擦系数为$\mu$)\\
			用转动定理\\
			$
			M=J\alpha\\
			M=J\frac{\ud \omega}{\ud t}\\
			M \ud t = J \ud \omega\\
			\int_{0}^{t}M\ud t = \int_{\omega_0}^{0}J \ud \omega\\
			M=fr\\
			\ud M = \ud f \cdot r\\
			\ud M = \mu g \ud m \cdot r\\
			\ud M = \mu g \sigma \ud s \cdot r\\
			\ud M = \mu g \sigma 2\pi r \ud r \cdot r\\
			M = \int \ud M\\
			M = \int_{0}^{R}\mu g \sigma 2\pi r \ud r \cdot r\\
			\int_{0}^{t}\ud t\int_{0}^{R}\mu g \sigma 2\pi r \ud r \cdot r = \int_{\omega_0}^{0}J \ud \omega\\
			$\\
			$
			M=J\alpha\\
			M=J\frac{\ud \omega}{\ud t}\\
			M \ud t = J \ud \omega\\
			\int_{0}^{t}M\ud t = \int_{\omega_0}^{0}J \ud \omega\\
			\int_{0}^{t}\int \ud M\ud t = \int_{\omega_0}^{0}J \ud \omega\\
			\int_{0}^{t}\int \ud f \cdot r\ud t = \int_{\omega_0}^{0}J \ud \omega\\
			\int_{0}^{t}\int \mu g \ud m \cdot r\ud t = \int_{\omega_0}^{0}J \ud \omega\\
			\int_{0}^{t}\int \mu g \sigma \ud s \cdot r\ud t = \int_{\omega_0}^{0}J \ud \omega\\
			\int_{0}^{t}\int \mu g \sigma 2\pi r \ud r \cdot r\ud t = \int_{\omega_0}^{0}J \ud \omega\\	
			\int_{0}^{t}\int_{0}^{R} \mu g \sigma 2\pi r \ud r \cdot r\ud t = \int_{\omega_0}^{0}J \ud \omega\\	
			\int_{0}^{t}\ud t\int_{0}^{R}\mu g \sigma 2\pi r \ud r \cdot r = \int_{\omega_0}^{0}J \ud \omega\\
			$\\\\
			$
			M=J\alpha\\
			M=J\frac{\ud \omega}{\ud t}\\
			M \ud t = J \ud \omega\\
			\int_{0}^{t}M\ud t = \int_{\omega_0}^{0}J \ud \omega\\
			\int_{0}^{t}\left(\int \ud M\right)\ud t = \int_{\omega_0}^{0}J \ud \omega\\
			\int_{0}^{t}\left(\int \ud f \cdot r\right)\ud t = \int_{\omega_0}^{0}J \ud \omega\\
			\int_{0}^{t}\left(\int \mu g \ud m \cdot r\right)\ud t = \int_{\omega_0}^{0}J \ud \omega\\
			\int_{0}^{t}\left(\int \mu g \sigma \ud s \cdot r\right)\ud t = \int_{\omega_0}^{0}J \ud \omega\\
			\int_{0}^{t}\left(\int \mu g \sigma 2\pi r \ud r \cdot r\right)\ud t = \int_{\omega_0}^{0}J \ud \omega\\	
			\int_{0}^{t}\left(\int_{0}^{R} \mu g \sigma 2\pi r \ud r \cdot r\right)\ud t = \int_{\omega_0}^{0}J \ud \omega\\
			\int_{0}^{t}\left(\int_{0}^{R} \mu g \frac{m}{\pi R^2} 2\pi r \ud r \cdot r\right)\ud t = \int_{\omega_0}^{0}J \ud \omega\\
			\int_{0}^{t}\left(\int_{0}^{R} \mu g \frac{m}{\pi R^2} 2\pi r \ud r \cdot r\right)\ud t = \int_{\omega_0}^{0}\frac{1}{2}m R^2 \ud \omega\\	
			$
			\begin{eqnarray*}
			M& = &J\alpha\\
			M& = &J\frac{\ud \omega}{\ud t}\\
			M \ud t & = & J \ud \omega\\
			\int_{0}^{t}M\ud t & = & \int_{\omega_0}^{0}J \ud \omega\\
			\int_{0}^{t}\left(\int \ud M\right)\ud t & = & \int_{\omega_0}^{0}J \ud \omega\\
			\int_{0}^{t}\left(\int \ud f \cdot r\right)\ud t & = & \int_{\omega_0}^{0}J \ud \omega\\
			\int_{0}^{t}\left(\int \mu g \ud m \cdot r\right)\ud t & = & \int_{\omega_0}^{0}J \ud \omega\\
			\int_{0}^{t}\left(\int \mu g \sigma \ud s \cdot r\right)\ud t & = & \int_{\omega_0}^{0}J \ud \omega\\
			\int_{0}^{t}\left(\int \mu g \sigma 2\pi r \ud r \cdot r\right)\ud t & = & \int_{\omega_0}^{0}J \ud \omega\\	
			\int_{0}^{t}\left(\int_{0}^{R} \mu g \sigma 2\pi r \ud r \cdot r\right)\ud t & = & \int_{\omega_0}^{0}J \ud \omega\\
			\int_{0}^{t}\left(\int_{0}^{R} \mu g \frac{m}{\pi R^2} 2\pi r \ud r \cdot r\right)\ud t & = & \int_{\omega_0}^{0}J \ud \omega\\
			\int_{0}^{t}\left(\int_{0}^{R} \mu g \frac{m}{\pi R^2} 2\pi r \ud r \cdot r\right)\ud t & = & \int_{\omega_0}^{0}\frac{1}{2}m R^2 \ud \omega\\	
			\end{eqnarray*}
		\end{enumerate}

		\section{title}
		\subsection{title}
		$\displaystyle\\
		\Delta \varphi \rightarrow \Delta x:\\
		y=A\cos \left[\omega(t-\frac{x}{u})+\varphi\right]\\
		\Phi = \omega(t-\frac{x}{u})+\varphi\\
		\Delta \Phi = \left[\omega(t-\frac{x_1}{u})+\varphi\right] - \left[\omega(t-\frac{x_2}{u})+\varphi\right]
		$\\
		同一列波初相相同,t相同\\
		$\displaystyle
		\Delta \Phi = \omega\left(\frac{x_1-x_2}{u}\right) = \frac{2\pi}{T}\left(\frac{x_1-x_2}{u}\right) = 2\pi\frac{x_1-x_2}{\lambda}
		$

	\end{Large}

\end{document}
