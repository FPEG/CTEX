\documentclass[UTF8,a4paper,12pt,scheme=chinese]{ctexbook}

%\setlength{\textwidth}{550pt}
%\setlength{\hoffset}{-1.2in}
%\setlength{\voffset}{-1in}
\newlength{\chaptertopskip}
\setlength{\chaptertopskip}{10pt}

\usepackage{amsmath}
\usepackage{textcomp} 
\usepackage{graphicx}
\usepackage{xcolor}
\usepackage{amsthm}
\usepackage{longtable}
\usepackage{makecell}
%\usepackage{slashbox}
\usepackage{diagbox}
\newcommand{\hlx}[2]{%
	\IfEqCase{#1}{%
		{1}{
			\colorbox{yellow!50}{$\displaystyle#2$}
		}
		{b}{
			\colorbox{yellow!50}{$\scriptstyle#2$}
		}
		{3}{
			\colorbox{yellow!50}{$\scriptscriptstyle#2$}
		}
	}
		% you can add more cases here as desired
	
}%


\newcommand{\ud}{\mathrm{d}}
\newcommand{\lc}{\left(}
\newcommand{\rc}{\right)}
\newcommand{\jz}[1]{\boldsymbol{#1}}

\newcommand{\hl}[1]{\colorbox{yellow}{#1}}
\newcommand{\hla}[1]{%
	\colorbox{yellow!50}{$\displaystyle#1$}}
\newcommand{\hlb}[1]{%
	\colorbox{yellow!50}{$\scriptstyle#1$}}
\newcommand{\hlc}[1]{%
	\colorbox{yellow!50}{$\scriptscriptstyle#1$}}

%\makeatletter
%\DeclareFontFamily{U}{tipa}{}
%\DeclareFontShape{U}{tipa}{m}{n}{<->tipa10}{}
%\newcommand{\arc@char}{{\usefont{U}{tipa}{m}{n}\symbol{62}}}%
%
%\newcommand{\arc}[1]{\mathpalette\arc@arc{#1}}
%
%\newcommand{\arc@arc}[2]{%
%	\sbox0{$\m@th#1#2$}%
%	\vbox{
%		\hbox{\resizebox{\wd0}{\height}{\arc@char}}
%		\nointerlineskip
%		\box0
%	}%
%}
\everymath{\displaystyle}
\newtheorem{theorem}{定理}[section]
\newtheorem*{theorem*}{非书定理}
%\theoremstyle{plain}
\newtheorem{definition}{定义}[section]

\makeatother

\begin{document}
	\chapter{行列式}
	\chapter{矩阵及其运算}
	\section{矩阵的幂}
	\begin{longtable}{|c|c|c|c|}
		\hline 
		 & 转置 & 行列式 & 逆 
		\\\hline  
		$ \jz{A}^T $ & $ (\jz{A}^T)^T=A $ & $ |\jz{A}^T|=|\jz{A}| $ & $ (\jz{A}^T)^{-1} = (\jz{A}^{-1})^T $ 
		\\\hline 
		$ |\jz{A}| $ & 无  & 无 & $ |\jz{A}|^{-1}=|\jz{A}^{-1}| $
		\\\hline
		$ \jz{A}^{-1} $ & $ (\jz{A}^{-1})^T = (\jz{A}^T)^{-1} $ & $ |\jz{A}^{-1}|=|\jz{A}|^{-1} $ & $ (\jz{A}^{-1})^{-1}=\jz{A} $
		\\\hline
		$ \jz{A}\pm\jz{B} $ & $ (\jz{A}\pm\jz{B})^T=\jz{A}^T\pm\jz{B}^T $ & 无 & 无
		\\\hline
		$ \lambda\jz{A} $ & $ (\lambda\jz{A})^T=\lambda\jz{A}^T $ & $ |\lambda\jz{A}|=\lambda^n|\jz{A}| $ & $ (\lambda\jz{A})^{-1}=\frac{1}{\lambda}\jz{A}^{-1} $
		\\\hline
		$ \jz{A}\jz{B} $ & $ (\jz{A}\jz{B})^T=\jz{B}^T\jz{A}^T $ &
		\makecell{$ |\jz{A}\jz{B}|=|\jz{B}\jz{A}|$\\$=|\jz{A}||\jz{B}|=|\jz{B}||\jz{A}| $} 
		& $ (\jz{A}\jz{B})^{-1}=\jz{B}^{-1}\jz{A}^{-1} $
		\\\hline
		对称$ \jz{A}\jz{B} $ & \makecell{$ (\jz{A}\jz{B})^T=\jz{A}\jz{B}=\jz{B}\jz{A} $\\$ (\jz{A}\jz{B})^T=\jz{B}^T\jz{A}^T=\jz{A}^T\jz{B}^T $}&同上&同上
		\\\hline
		$ \jz{A}^* $ & $ (\jz{A}^*)^T=(\jz{A}^T)^* $ & $ |\jz{A}^*|=|\jz{A}|^{n-1} $ & $ (\jz{A}^*)^{-1}=(\jz{A}^{-1})^*=\frac{\jz{A}}{|\jz{A}|} $
		\\\hline
		$ \jz{\varLambda} $ &&&
		\\\hline
	\end{longtable}
		\begin{longtable}{|c|c|c|c|}
		\hline 
		& 伴随 & 秩 &  
		\\\hline  
		$ \jz{A}^T $ & $ (\jz{A}^T)^*=(\jz{A}^*)^T $ & $ R(\jz{A}) $  &   
		\\\hline 
		$ |\jz{A}| $ & 无 & 无 &  
		\\\hline
		$ \jz{A}^{-1} $ & $ (\jz{A}^{-1})^*=(\jz{A}^*)^{-1}$ & 满秩$ R(\jz{A}) $ &  
		\\\hline
		$ \jz{A}\pm\jz{B} $ & 无 & $ \leq R(\jz{A})\pm R(\jz{B}) $ &
		\\\hline
		$ \lambda\jz{A} $ & $ (\lambda\jz{A})^*=\lambda^{n-1}\jz{A}^* $ & $ R(\jz{A}) $ &
		\\\hline
		$ \jz{A}\jz{B} $ & $ (\jz{A}\jz{B})^*=\jz{B}^*\jz{A}^* $ & $ \leq \min\{R(\jz{A}),R(\jz{B})\} $ &
		\\\hline
		对称$ \jz{A}\jz{B} $ &  &  &  
		\\\hline
		$ \jz{A}^* $ & $ (\jz{A}^*)^*=|\jz{A}|^{n-2}\jz{A} $ & 讨论 &  
		\\\hline
		$ \jz{\varLambda} $ &&&
		\\\hline
	\end{longtable}	
	\begin{longtable}{|c|c|c|}
		\hline
		& 逆 & 幂\\
		\hline
		 $ \left[\begin{array}{cc}
		a&b\\
		c&d
		\end{array}\right]   $ & $ \hla{\frac{1}{a d-b c}}		
		\left(
		\begin{array}{cc}
		d & -b \\
		-c & a \\
		\end{array}
		\right) $ & 复杂\\
		\hline
		
		
	\end{longtable}
	\begin{enumerate}
		\item $\left(
		\begin{array}{cc}
		1 & 1 \\
		0 & 1 \\
		\end{array}
		\right)^n=\left(
		\begin{array}{cc}
		1 & n \\
		0 & 1 \\
		\end{array}
		\right)$
		\item $\left(
		\begin{array}{cc}
		1 & 0 \\
		1 & 1 \\
		\end{array}
		\right)^n=\left(
		\begin{array}{cc}
		1 & 0 \\
		n & 1 \\
		\end{array}
		\right)$
		\item $
		\left(
		\begin{array}{cc}
		a & 0 \\
		0 & b \\
		\end{array}
		\right)^n=
		\left(
		\begin{array}{cc}
		a^n & 0 \\
		0 & b^n \\
		\end{array}
		\right)
		$
		\item $
		\left(
		\begin{array}{ccc}
		0 & 0 & a \\
		0 & b & 0 \\
		c & 0 & 0 \\
		\end{array}
		\right)^{-1}
		=
		\left(
		\begin{array}{ccc}
		0 & 0 & a^{-1} \\
		0 & b^{-1} & 0 \\
		c^{-1} & 0 & 0 \\
		\end{array}
		\right)^T
		$\\
		如果对分块矩阵使用,需要先证分块矩阵可逆
		\item $
		\left(
		\begin{array}{cc}
		a & b \\
		c & d \\
		\end{array}
		\right)
		=		
		\hla{\frac{1}{a d-b c}}		
		\left(
		\begin{array}{cc}
		d & -b \\
		-c & a \\
		\end{array}
		\right)
		$
	\end{enumerate}
	\chapter{矩阵的初等变换与线性方程组}
	\section{矩阵的初等变换}
	\subsection{矩阵.初等行变换()}
%	\begin{definition}[初等行变换]~{}\\
%		
%		\begin{enumerate}
%			\item 对换两行
%			\item 一行乘以k
%			\item 倍加行
%		\end{enumerate}
%	\end{definition}
	\begin{enumerate}
		\item 重载\quad 对换两行
		\item 重载\quad 一行乘以k
		\item 重载\quad 行加行
	\end{enumerate}
	\subsection{矩阵.等价($ \boldsymbol{A},\boldsymbol{B} $)}
	\begin{itemize}
		\item 参数
		\begin{itemize}
			\item 矩阵$ \boldsymbol{A} $
			\item 矩阵$ \boldsymbol{B} $
		\end{itemize}
		\item 返回
		\begin{itemize}
			\item A等价B
			\begin{itemize}
				\item 条件\quad 矩阵.初等行变换($ \boldsymbol{A} $)=$ \boldsymbol{B} $
			\end{itemize}
			\item 否
		\end{itemize}
	\end{itemize}
	\subsection{矩阵.化为行阶梯形($ \boldsymbol{A} $)}
	\begin{definition}[行阶梯形矩阵,行最简形矩阵]
		
	\end{definition}
	\subsection{矩阵.化为行最简形($ \boldsymbol{A} $)}
	一对一关系
	\subsection{矩阵.化为标准形($ \boldsymbol{A} $)}
	\begin{theorem}[矩阵等价的充要条件]
		content...
	\end{theorem}
	\section{矩阵的秩}
	\chapter{向量组的线性相关性}
	\section{向量组及其线性组合}
	\subsection{n维向量}
	\begin{itemize}
		\item 定义:n个有次序的数所组成的数组
		\item 子类
		\begin{enumerate}
			\item 列向量 $\boldsymbol{a}$
			\item 行向量 $\boldsymbol{a}^T$
		\end{enumerate}
		\item 成员
		\subitem 分量 $a_i$:n个
	\end{itemize}
	\subsection{向量组}
	\begin{itemize}
		\item 定义:若干个同维数的向量所组成的集合
		\item 格式:$A:\boldsymbol{a}_1,\boldsymbol{a}_2,\dots,\boldsymbol{a}_n$
		\item 成员
		\subitem 向量 $\boldsymbol{a}_i$:若干个
		\item 方法
		\begin{enumerate}
			\item 表示成矩阵:$\boldsymbol{A} = (\boldsymbol{a}_1,\boldsymbol{a}_2,\dots,\boldsymbol{a}_n)$
			\item 线性组合
		\end{enumerate}
	\end{itemize}
	\subsection{线性组合}
	\begin{itemize}
		\item 参数
		\begin{itemize}
			\item 向量组$A$
			\item 系数$k_1,k_2,\dotsb,k_m$
		\end{itemize}
		\item 返回
		\begin{itemize}
			\item 向量$\boldsymbol{b}$
			\item 线性方程组$\boldsymbol{b}=k_1\boldsymbol{a}_1+k_2\boldsymbol{a}_2+\dots+k_n\boldsymbol{a}_n$
		\end{itemize}
	\end{itemize}
	\subsection{线性表示}
	\begin{enumerate}
		\item 重载:向量$\boldsymbol{b}$能由向量组$A$线性表示
		\begin{itemize}
			\item 参数
			\begin{itemize}
				\item 向量组$A$
				\item $\exists$系数$k_1,k_2,\dotsb,k_m$
				\item 向量$\boldsymbol{b}$
			\end{itemize}
			\item 返回
			\begin{itemize}
				\item 成立
				\begin{itemize}
					\item 条件:$\boldsymbol{b}=$线性组合$(A,k)$
					\item 等价
					\begin{enumerate}
						\item $\Leftrightarrow R(\boldsymbol{A})=R(\boldsymbol{A},\boldsymbol{b})$(定理1)
					\end{enumerate}
				\end{itemize}
				
					
				\item 不成立
			\end{itemize}
		\end{itemize}
		\item 重载:向量组$ B $能由向量组$A$线性表示
		\begin{itemize}
			\item 参数
			\begin{itemize}
				\item 向量组$A$
				\item $l$组系数$k_1,k_2,\dotsb,k_m$
				\item 向量组$B$
			\end{itemize}
			\item 返回
			\begin{itemize}
				\item 成立
				\begin{itemize}
					\item 条件:向量组$ B $中的每一项$\boldsymbol{b}_i=$线性组合$(A,(k_{1\dotsb m})_l)$
					\item 等价
					\begin{enumerate}
						\item $\Leftrightarrow\boldsymbol{b}=k_1\boldsymbol{a}_1+k_2\boldsymbol{a}_2+\dots+k_n\boldsymbol{a}_n,\exists k$
						\item $\Leftrightarrow\boldsymbol{b}=x_1\boldsymbol{a}_1+x_2\boldsymbol{a}_2+\dots+x_n\boldsymbol{a}_n$有解
						\item $\Leftrightarrow R(\boldsymbol{A})=R(\boldsymbol{A},\boldsymbol{b})$(定理2)
						\item $
						\left(\begin{array}{c c c| c c}
						1 & 0 & 0 & k_1 & j_1 \\
						0 & 1 & 0 & k_2 & j_2 \\
						0 & 0 & 1 & k_3 & j_3
						\end{array}\right)
						$
						$
						\left\{ \begin{array}{l}
						\boldsymbol{b}_1 = k_1\boldsymbol{a}_1+k_2\boldsymbol{a}_2+k_3\boldsymbol{a}_3\\
						\boldsymbol{b}_2 = j_1\boldsymbol{a}_1+j_2\boldsymbol{a}_2+j_3\boldsymbol{a}_3
						\end{array} \right.
						$
					\end{enumerate}
				\end{itemize}
				\item 不成立
			\end{itemize}
		\end{itemize}
	\end{enumerate}
		
	\subsection{向量组等价}
	\begin{itemize}
		\item 格式:$\boldsymbol{A}\Leftrightarrow\boldsymbol{B}$
		\item 参数
		\begin{itemize}
			\item 向量组$A$
			\item 向量组$B$
		\end{itemize}
		\item 返回
		\begin{itemize}
			\item 成立
			\begin{itemize}
				\item 条件
				\subitem 线性表示$(A,k_{1\dotsb m},B)=$成立
				\subitem and
				\subitem 线性表示$(B,k_{1\dotsb m},A)=$成立
				\item 等价
				\begin{enumerate}
					\item $\Leftrightarrow R(\boldsymbol{A})=R(\boldsymbol{B})=R(\boldsymbol{A},\boldsymbol{B})$(定理2推论)
				\end{enumerate}
			\end{itemize}
			
			\item 不成立
		\end{itemize}
	\end{itemize}
	

	\section{向量组的线性相关性}~\\
	\subsection{判断线性相关}
	\begin{itemize}
		\item 参数
		\begin{itemize}
			\item 向量组$A$
			\item $\exists$系数$k_1,k_2,\dotsb,k_m$
		\end{itemize}
		\item 返回
		\begin{itemize}
			\item 线性相关
			\begin{itemize}
				\item 条件
				\subitem $k_1,k_2,\dotsb,k_m$,$0=k_1\boldsymbol{a}_1+k_2\boldsymbol{a}_2+\dots+k_n\boldsymbol{a}_n$
				\item 等价
				\begin{enumerate}
					\item $\Leftrightarrow k_1\boldsymbol{a}_1+k_2\boldsymbol{a}_2+\dots+k_n\boldsymbol{a}_n=0,\exists k$不全为0
					\item$\Leftrightarrow x_1\boldsymbol{a}_1+x_2\boldsymbol{a}_2+\dots+x_n\boldsymbol{a}_n=0$有非零解
					\item$\Leftrightarrow R(\boldsymbol{A})<$向量个数$m$(定理4)
					\item$\Leftrightarrow$(方阵)$|A|\ne 0$
				\end{enumerate}
			\end{itemize}
			\item 线性无关
		\end{itemize}
	\end{itemize}
	\subsection{线性相关}
	\begin{itemize}
		\item 性质
		\begin{enumerate}
			\item 有关+1=有关(定理5)
			\item 无关-1=无关(定理5)
			\item 无关+b=有关,则b可用无关唯一表示(定理5)
		\end{enumerate}
	\end{itemize}
	\subsection{0的妙用}
	\begin{enumerate}
		\item $|A|\ne 0$
		\item$\Leftrightarrow A$可逆
		\item$\Leftrightarrow A$满秩
		\item$\Leftrightarrow A$非退化
		\item$\Leftrightarrow A$非奇异
		\item$\Leftrightarrow Ax=B$有唯一解
		\item$\Leftrightarrow Ax=0$无非0解
		\item$\Leftrightarrow \boldsymbol{A}$的向量组线性无关
	\end{enumerate}
	
	\section{向量组的秩}
	\subsection{定义5}
	向量组$A$.秩=向量组$A$.最大线性无关向量组.向量个数
	\subsection{最大线性无关向量组.定义}
	(定义5推论)
	\begin{itemize}
		\item 
	\end{itemize}
	\subsection{矩阵.秩.性质}
	\begin{itemize}
		\item 等价:矩阵.列向量组.秩,矩阵.行向量组.秩
	\end{itemize}
	\section{线性方程组解的结构}
	\subsection{齐次线性方程组.解向量}
	\begin{itemize}
		\item 定义\quad方程组的解x的集合
		\item 性质
		\begin{enumerate}
			\item 可加性\quad$ x=\xi_1+\xi_2 $为方程解
			\item 可倍性\quad$ x=k\xi_1 $为方程解
		\end{enumerate}
	\end{itemize}
	\subsection{齐次线性方程组.基础解系}
	\begin{itemize}
		\item 定义\quad 齐次线性方程组.解集.最大线性无关组
	\end{itemize}
	\subsection{齐次线性方程组.解集.求秩(n,r)}
	\begin{itemize}
		\item 参数
		\begin{itemize}
			\item 齐次线性方程组.元数n
			\item 齐次线性方程组.秩r
		\end{itemize}
		\item 返回
		\subitem n-r
	\end{itemize}
	\subsection{齐次线性方程组.求基础解系(A)}
	\begin{itemize}
		\item 参数
		\item 过程
		\begin{enumerate}
			\item 矩阵.化为行阶梯形($ \boldsymbol{A} $)
			\item R=齐次线性方程组.解集.求秩(n,r)
			\item 挑出R个非首列列向量
			\item 取R个列向量的未知数组成单位矩阵
			\item 单位矩阵代回方程算出其他未知数
		\end{enumerate}
	\end{itemize}
\end{document}
