\documentclass[UTF8,a4paper,12pt,scheme=chinese]{ctexbook}

\addtolength{\textwidth}{1in}
\addtolength{\hoffset}{-0.5in}
%\setlength{\voffset}{-1in}

\usepackage{amsmath}
\usepackage{amssymb}
\usepackage{textcomp} 
\usepackage{graphicx}
\usepackage{xcolor}
\usepackage{mathrsfs}
\usepackage{tikz}
\usepackage{amsthm}
%\usepackage{soul}
\usetikzlibrary{arrows,decorations.pathmorphing,backgrounds,positioning,fit,petri}

%\usepackage{latexsym}

\newcommand{\hlx}[2]{%
	\IfEqCase{#1}{%
		{1}{
			\colorbox{yellow!50}{$\displaystyle#2$}
		}
		{b}{
			\colorbox{yellow!50}{$\scriptstyle#2$}
		}
		{3}{
			\colorbox{yellow!50}{$\scriptscriptstyle#2$}
		}
	}
	% you can add more cases here as desired
	
}%


%\pagestyle{headings}
%\ctexset{

%}
\pagestyle{headings}
\ctexset{
	%	section={	
	%		name={第,章},
	%		number=\arabic{section},
	%	},
	section={
		name={\S,},
		number=\arabic{section},
	},
	subsection={
		name={,},
		number=\arabic{subsection}.,
	},
	subsubsection/name = {},
	subsubsection={
		%		name={asd,啊},
		number=\arabic{subsubsection}.,
		aftername={},
		format=\Large\bfseries\CJKfamily{zhsong},
	},
	paragraph={
		%		name={asd,啊},
		%		number=\arabic{subsubsection},
		format=\bfseries\youyuan,
	},
}
\setcounter{secnumdepth}{3}
\newcommand{\ud}{\mathrm{d}}
\newcommand{\Arg}{\mathrm{Arg}}
\newcommand{\lc}{\left(}
\newcommand{\rc}{\right)}

\newcommand{\hl}[1]{\colorbox{yellow}{#1}}
\newcommand{\hla}[1]{%
	\colorbox{yellow!50}{$\displaystyle#1$}}
\newcommand{\hlb}[1]{%
	\colorbox{yellow!50}{$\scriptstyle#1$}}
\newcommand{\hlc}[1]{%
	\colorbox{yellow!50}{$\scriptscriptstyle#1$}}

\makeatletter
\DeclareFontFamily{U}{tipa}{}
\DeclareFontShape{U}{tipa}{m}{n}{<->tipa10}{}
\newcommand{\arc@char}{{\usefont{U}{tipa}{m}{n}\symbol{62}}}%

\newcommand{\arc}[1]{\mathpalette\arc@arc{#1}}

\newcommand{\arc@arc}[2]{%
	\sbox0{$\m@th#1#2$}%
	\vbox{
		\hbox{\resizebox{\wd0}{\height}{\arc@char}}
		\nointerlineskip
		\box0
	}%
}

\everymath{\displaystyle}
%\newtheorem{btheorem}{b定理}[section]
\newtheorem{theorem}{定理}[chapter]
%\renewcommand{\theorem}{\arabic{subsection}.\arabic{thm}}
\newtheorem*{theorem*}{非书定理}
\theoremstyle{plain}
\newtheorem{definition}{定义}[section]
\newtheorem{property}{性质}[subsection]
\makeatother

\begin{document}
	\chapter{复数与复变函数}
	\section{复数及其代数运算}
	\subsection{复数的概念}
	\subsection{复数的代数运算}
	\paragraph{共轭复数}$\left\lbrace
	\begin{array}{l} \overline{z_1\pm z_2}=\overline{z_1}\pm\overline{z_2}\\\overline{z_1\cdot z_2}=\overline{z_1}\cdot\overline{z_2}
	\end{array}
	\right. $
	\section{复数的几何表示}
	\subsection{复平面}
	\paragraph{绝对值、模} $\left\lbrace
	\begin{array}{l} |z|=r=\sqrt{x(a)^2+y(b)^2}\\z\overline{z}=|z|^2=|z^2|
	\end{array}
	\right. $
	\paragraph{幅角}$ \Arg z=\theta $ , $ \tan(\Arg z) = \dfrac{y}{x} $
	\paragraph{主值}$ \arg z=\theta_0(-\pi<\Arg z\leqslant\pi) $
	\paragraph{加减}与向量一致
	\paragraph{三角表达式}$ z=r(\cos\theta+i\sin\theta) $
	\paragraph{欧拉公式}$ e^{i\theta}=\cos\theta+i\sin\theta $
	\paragraph{指数表达式}$ z=re^{i\theta} $
	\paragraph{其他证明}$\left\lbrace
	\begin{array}{l} |z_1\overline{z_2}|=|z_1||z_2|\\
	|z_1+z_2|\leqslant|z_1|+|z_2|
	\end{array}
	\right. $
	\subsection{复球面}
	\paragraph{北极}在上面
	\paragraph{复球面}
	\paragraph{扩充复平面}包含无穷远点
	\section{复数的乘幂与方根}
	\subsection{乘积与商}
	\begin{theorem}
		$\left\lbrace
		\begin{array}{l} 
		|z_1z_2|=|z_1||z_2|\\
		\Arg(z_1z_2)=\Arg(z_1)+\Arg(z_2)\\
		z_1z_2=r_1r_2e^{i(\theta_2+\theta_1)}
		\end{array}
		\right. $
	\end{theorem}
	\paragraph{乘法几何意义}乘以长度并逆时针旋转
	\begin{theorem}
		$\left\lbrace
		\begin{array}{l} 
		\left| \dfrac{z_2}{z_1}\right| =\dfrac{|z_2|}{|z_1|}\\
		\Arg\left( \dfrac{z_2}{z_1}\right) =\Arg(z_2)-\Arg(z_1)\\
		\dfrac{z_2}{z_1}=\dfrac{r_2}{r_1}e^{i(\theta_2-\theta_1)}
		\end{array}
		\right. $
	\end{theorem}
	\paragraph{除法几何意义}除以长度并顺时针旋转
	\subsection{幂与根}
	\paragraph{n次幂}$ z^n=r^n(\cos n\theta+i\sin n\theta) $
	\paragraph{棣莫弗公式}$ (\cos\theta+i\sin\theta)^n=\cos n\theta+i\sin n\theta $
	\paragraph{n次根}$ s$
	\section{区域}
	\section{复变函数}
	\section{复变函数的极限和连续性}
	\chapter{解析函数}
	\section{解析函数的概念}
	
\end{document}