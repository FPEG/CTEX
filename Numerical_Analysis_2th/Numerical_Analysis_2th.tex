\documentclass[UTF8,a4paper,12pt,scheme=chinese]{ctexbook}

\addtolength{\textwidth}{1in}
\addtolength{\hoffset}{-0.5in}
%\setlength{\voffset}{-1in}

\usepackage{amsmath}
\usepackage{amssymb}
\usepackage{textcomp} 
\usepackage{graphicx}
\usepackage{xcolor}
\usepackage{mathrsfs}
\usepackage{tikz}
\usepackage{amsthm}
%\usepackage{soul}
\usetikzlibrary{arrows,decorations.pathmorphing,backgrounds,positioning,fit,petri}

%\usepackage{latexsym}

\newcommand{\hlx}[2]{%
	\IfEqCase{#1}{%
		{1}{
			\colorbox{yellow!50}{$\displaystyle#2$}
		}
		{b}{
			\colorbox{yellow!50}{$\scriptstyle#2$}
		}
		{3}{
			\colorbox{yellow!50}{$\scriptscriptstyle#2$}
		}
	}
	% you can add more cases here as desired
	
}%


%\pagestyle{headings}
%\ctexset{

%}
\pagestyle{headings}
\ctexset{
	%	section={	
	%		name={第,章},
	%		number=\arabic{section},
	%	},
	section={
		name={\S,},
		number=\arabic{section},
	},
	subsection={
		name={,},
		number=\arabic{subsection}.,
	},
	subsubsection/name = {},
	subsubsection={
		%		name={asd,啊},
		number=\arabic{subsubsection}.,
		aftername={},
		format=\Large\bfseries\CJKfamily{zhsong},
	},
	paragraph={
		%		name={asd,啊},
		%		number=\arabic{subsubsection},
		format=\bfseries\youyuan,
	},
}
\setcounter{secnumdepth}{3}
\newcommand{\ud}{\mathrm{d}}
\newcommand{\Arg}{\mathrm{Arg}}
\newcommand{\lc}{\left(}
\newcommand{\rc}{\right)}

\newcommand{\hl}[1]{\colorbox{yellow}{#1}}
\newcommand{\hla}[1]{%
	\colorbox{yellow!50}{$\displaystyle#1$}}
\newcommand{\hlb}[1]{%
	\colorbox{yellow!50}{$\scriptstyle#1$}}
\newcommand{\hlc}[1]{%
	\colorbox{yellow!50}{$\scriptscriptstyle#1$}}

\makeatletter
\DeclareFontFamily{U}{tipa}{}
\DeclareFontShape{U}{tipa}{m}{n}{<->tipa10}{}
\newcommand{\arc@char}{{\usefont{U}{tipa}{m}{n}\symbol{62}}}%

\newcommand{\arc}[1]{\mathpalette\arc@arc{#1}}

\newcommand{\arc@arc}[2]{%
	\sbox0{$\m@th#1#2$}%
	\vbox{
		\hbox{\resizebox{\wd0}{\height}{\arc@char}}
		\nointerlineskip
		\box0
	}%
}

\everymath{\displaystyle}
%\newtheorem{btheorem}{b定理}[section]
\newtheorem{theorem}{定理}[chapter]
%\renewcommand{\theorem}{\arabic{subsection}.\arabic{thm}}
\newtheorem*{theorem*}{非书定理}
\theoremstyle{plain}
\newtheorem{definition}{定义}[section]
\newtheorem{property}{性质}[subsection]
\makeatother

\begin{document}
	\chapter{求解方程}
	\section{二分法}
	\section{不动点迭代}
	\section{精度的极限}
	\section{牛顿方法}
	\subsection{牛顿方法的二次收敛}
	\paragraph{证明}
	\subsubsection{牛顿下山法}
	\subsubsection{简化牛顿迭代法}
	\paragraph{公式}$
	\begin{array}{l}
	x_{i+1}=x_i-\dfrac{f(x_i)}{f(x_0)}\\
	x_{i+1}=x_i-\dfrac{f(x_i)}{C}\\
	\varphi(x_{i})=x_i-\dfrac{f(x_i)}{C}\\
	\varphi'(x^*)=1-\dfrac{f'(x^*)}{C}\mbox{线性收敛}\\
	\end{array} 
	$
	\subsubsection{带参数的你的的地方(m重根)}
	\paragraph{title}$  $
	\begin{definition}[m重根]
		$ f(x)=(x-x^*)^m\cdot h(x),h'(x)\neq0 $\\
		$ g(x)=[f(x)]^{\dfrac{1}{m(*)}}=(x-x^*)^m\cdot h(x),h'(x)\neq0 $
		
	\end{definition}
	\chapter{方程组}
	\section{高斯消去法}
	防止误差
	\subsection{顺序高斯消元/朴素高斯消去}
	有误差,淹没现象,大乘子,
	$ -\dfrac{a_{21}}{a_{11}},a_{11} $太小
	\subsection{(列)主元高斯消元}
	选择绝对值最大的数做主元
	\section{LU分解}
	顺序消元法加强版\\
	$ \left[\begin{array}{cccc}
	a_{11}&\cdots&a_{1n}&b_1\\
	0&&a_{nn}&b_n
	\end{array}\right] L'A=U,A=(L')^{-1}U=LU$
	\paragraph{title}$AX=b$
	\paragraph{L,b已知,求C}$L(UX)=b=LC$
	\paragraph{求X}$ UX=C $
	
	\paragraph{A化为LU的步骤}
	A化为上三角:初等行变换=左乘(初等矩阵)
	\paragraph{title}$ \left[\begin{array}{ccccc}
	1\\
	-C_{21}&1\\
	&&1\\
	&&&1\\
	&&&&1
	\end{array}\right]
	\left[\begin{array}{ccccc}
	1\\
	&1\\
	-C_{31}&&1\\
	&&&1\\
	&&&&1
	\end{array}\right]
	=
		\left[\begin{array}{ccccc}
	1\\
	-C_{21}&1\\
	-C_{31}&&1\\
	&&&1\\
	&&&&1
	\end{array}\right]
	=L_1
	 $
	 \paragraph{L逆}
	 $ \left[\begin{array}{ccccc}
	 1\\
	 -C_{21}&1\\
	 &&1\\
	 &&&1\\
	 &&&&1
	 \end{array}\right]
	 \left[\begin{array}{ccccc}
	 1\\
	 C_{21}&1\\
	 &&1\\
	 &&&1\\
	 &&&&1
	 \end{array}\right]
	 =
	 \left[\begin{array}{ccccc}
	 1\\
	 &1\\
	 &&1\\
	 &&&1\\
	 &&&&1
	 \end{array}\right] $其他都一样
	\section{误差来源}
	\section{PA=LU分解}
	
\end{document}













